% ---
% RESUMOS
% ---

% resumo em português
\setlength{\absparsep}{18pt} % ajusta o espaçamento dos parágrafos do resumo
\begin{resumo}
%Elemento obrigatório que contém a apresentação concisa dos pontos relevantes do trabalho, fornecendo uma visão rápida e clara do conteúdo e das conclusões do mesmo. A apresentação e a redação do resumo devem seguir os requisitos estipulados pela NBR 6028 (ABNT, 2003). Deve descrever de forma clara e sintética a natureza do trabalho, o objetivo, o método, os resultados e as conclusões, visando fornecer elementos para o leitor decidir sobre a consulta do trabalho no todo.

%A participação feminina e evasão nos cursos de graduação em ciência, tecnologia, engenharia e matemática continuam sendo um problema mundial e historicamente conhecido. No Brasil o combate a esta desigualdade não parte da legislação, porém muitos projetos e programas são iniciados na academia, como é o exemplo do programa Meninas Digitais. A democratização da educação e a inclusão das mulheres nestas áreas é sinônimo de desenvolvimento e entende-se que o combate a desigualdade feminina inicia-se com sua presença nas universidades. Com isso, tem-se como  objetivo analisar a presença das mulheres nos cursos de TIC em universidades brasileiras e sua possível relação com as ações dos projetos parceiros do programa Meninas Digitais. Visando atingir este objetivo, inicia-se uma investigação sobre o tema evasão e suas relações com gênero, uma exploração das ações realizadas para diminuir a evasão das mulheres na área, realiza-se um mapeamento sistemático da literatura sobre evasão escolar, a coleta e processamento da base de dados relacionada a evasão e gênero e análises gráficas destas relações, além disso, traz-se a proposta de uma entrevista com os projetos parceiros do programa Meninas Digitais.


A baixa representatividade feminina e a questão da evasão nos cursos de graduação em ciência, tecnologia, engenharia e matemática continuam sendo um problema mundial e historicamente conhecidos. Com o intuito de minimizar estes problemas muitos projetos e programas são realizados, como o Programa Meninas Digitais. A democratização da educação e a inclusão das mulheres nestas áreas são fundamentais para o desenvolvimento de uma sociedade e entende-se que o combate a desigualdade feminina inicia-se com sua presença nas universidades. Desta forma, este trabalho tem como objetivo analisar a presença das mulheres nos cursos de TIC em universidades brasileiras e também identificar uma possível influência com as ações dos projetos parceiros do programa Meninas Digitais. Foi realizado um mapeamento sistemático da literatura sobre evasão escolar, a coleta e processamento da base de dados relacionada a evasão e gênero e análises gráficas destas relações. 

 \textbf{Palavras-chave}: 1. Computação. 2. Evasão Escolar. 3. Meninas Digitais. 4. Mulheres. 5. INEP.
\end{resumo}
