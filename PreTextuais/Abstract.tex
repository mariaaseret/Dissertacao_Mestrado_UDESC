% ---
% Abstract
% ---

% resumo em inglês
\begin{resumo}[Abstract]
 \begin{otherlanguage*}{english}
  % Elemento obrigatório para todos os trabalhos de conclusão de curso. Opcional para os demais trabalhos acadêmicos, inclusive para artigo científico. Constitui a versão do resumo em português para um idioma de divulgação internacional. Deve aparecer em página distinta e seguindo a mesma formatação do resumo em português.

The low representation of women and the issue of dropout in undergraduate courses in science, technology, engineering and mathematics continue to be a worldwide and historically known problem. In order to minimize these problems, many projects and programs are carried out, such as the Meninas Digitais Program. The democratization of education and the inclusion of women in these areas are fundamental for the development of a society and it is understood that the fight against female inequality begins with their presence in universities. This work aims to analyze the presence of women in ICT courses in brazilian universities and also to identify a possible influence with the actions of partner projects of the Meninas Digitais Program. A systematic mapping on dropout was carried out, the collection and processing of the database related to dropout and gender and graphic analyzes of these relationships were carried out.

   \textbf{Keywords}: 1. Computing. 2. Dropout. 3. Meninas Digitais. 4. Women. 5. INEP.
 \end{otherlanguage*}
\end{resumo}
