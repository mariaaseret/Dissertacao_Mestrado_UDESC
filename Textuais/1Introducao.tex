\chapter{Introdução}\label{cap:Introducao}
% No presente capítulo apresenta-se a definição do problema, a justificativa para o estudo, os objetivos gerais e específicos, a metodologia utilizada e a estruturação.

% \section{Definição do Problema}\label{sec:DefinicaoProblema}


%A mudança das tecnologias traz uma grande demanda de profissionais das Áreas de ciência, tecnologia, engenharia e matemática, do inglês \textit{science, technology, engineering and mathematics} (STEM).

%Como área de tecnologia, entende-se ciência, tecnologia, engenharia e matemática, do inglês \textit{science, technology, engineering and mathematics} (STEM) e 
%ta quasw

A participação feminina e a evasão escolar nos cursos de graduação em  ciência, tecnologia, engenharia e matemática, do inglês \textit{Science, Technology, Engineering and Mathematics} (STEM) são um problema mundial e historicamente conhecidos. 

% Até início do século XX a presença feminina era considerada indesejada nas universidades \cite{barros:2020}.

A partir disso, a Organização das Nações Unidas (ONU) definiu dentro de seus pilares o quinto Objetivo de Desenvolvimento Sustentável, a inclusão de mulheres nas áreas de STEM a partir da igualdade de gênero.
Apesar das mulheres terem alcançado maior escolaridade que os homens a nível mundial e também dos esforços significativos com as iniciativas públicas, privadas e da sociedade civil para a equidade de gênero, a ONU relata que a maior parte das mulheres está fora da carreira das STEM e com  remunerações inferiores. Em números, a América Latina está melhor representada, 45,1\% de trabalhadores nas STEM são mulheres \cite{unesco:2019}, porém o mesmo não ocorre no Brasil, onde apenas 24\% dos trabalhadores são mulheres desta área \cite{fernandes:2021}. 

Como combate as desigualdades, a inclusão de mulheres nas STEM constitui um dos pilares da ONU para o objetivo número 5 da igualdade de gênero, pois apesar da maior escolaridade comparada a dos homens, elas ainda recebem menores salários no mercado de trabalho. E mesmo quando as mulheres escolhem a carreira das STEM continuam com menores remunerações para os mesmos cargos.

As motivações para essa desigualdade são muitas vezes culturais, como relata \citeonline{alfred:2019} que expõem o senso comum de que mulheres não tem aptidão para as ciências duras, principalmente ciência da computação e as engenharias. Na mesma pesquisa, relata-se que no caso de alunas negras a discriminação institucional acontece ao longo de toda a vida, com punições disciplinares, notas menores e maiores evasões acadêmicas.

\citeonline{natansohn:2021} apresentam que a diversidade no meio das STEM é papel importante, e sua falta, não apenas de gênero, mas também em um contexto das diversidades sociais tornam o ambiente desfavorável a serem ocupados por grupos subalternizados, como mulheres negras, indígenas e a comunidade LGBTQIA+.


No Brasil, o combate a esta desigualdade não parte da legislação, como aponta \citeonline{iwamoto:2022}, que mostra que estes incentivos da inclusão de cidadãos nas STEM é escasso e no caso da inclusão de mulheres é raríssimo. Ao mesmo tempo, a mesma autora aponta diversos projetos internacionais e nacionais, geralmente provindos pela academia para este incentivo.

O desenvolvimento do país está diretamente ligado a esta diversidade que se dá inicialmente com a presença feminina nas universidades e nos cursos das STEM, como aponta \citeonline{lee:2010}, que afirma que a democratização da educação e a inclusão das mulheres na STEM foi a principal fonte para o desenvolvimento da Coreia do Sul a partir da década de 1960.

Com o exposto, entende-se que o combate a desigualdade feminina nas STEM, de maneira geral, inicia-se com sua presença nas universidades e estudar, expor e analisar os fatores que acercam é uma necessidade.

Dada a relevância da análise da presença de mulheres nos cursos das STEM, o presente trabalho busca compreender e analisar a presença de mulheres nos cursos de computação e tecnologia de informação relacionando com os programas de incentivo a sua permanência ou ingresso nas universidades, neste caso o programa Meninas Digitais que será melhor apresentado na Seção \ref{sec:ProjetosLevantados}.

É importante ressaltar que a análise de gênero presente nessa dissertação é relacionada apenas com a classificação dos dados do Instituto Nacional de Estudos e Pesquisas Educacionais Anísio Teixeira (INEP), que não levam em conta a diversidade e pluralidade dos gêneros e optam por classificações binárias.



%o autor tal argumenta...

%Dada a relevância da análise da presença de mulheres nos cursos de Computação e Tecnologias da Informação e Comunicação%...

%\section{Questões de pesquisa}

\section{Justificativa}\label{sec:Justificativa}

A presença de mulheres e a diversidade de gênero nos cursos de Computação e Tecnologia da Informação e Comunicação segundo \citeonline{sbc:2021} são a tradução do que é o desenvolvimento sustentável, criativo e eficiente.
\citeonline{sbc:2021} também apontam sete
motivos (chamados de 7P’s) para promoção da presença diversa no meio da Computação e Tecnologias da Informação e Comunicação (TIC). Produtividade, Pioneirismo, Pertencimento, Parceria, Praticidade, Pluralidade e Persistência. No entanto apenas a equidade de gênero não é suficiente para trazer inovações e descobertas científicas coletivas, é necessário também implantar políticas efetivas e cuidadosamente pensadas.%MAIS DETALHES ** 

Com isso, a contribuição científica que se busca com esta pesquisa está na compreensão do cenário da presença das mulheres nos cursos de TIC e suas relações com os projetos de incentivo a permanência, trazendo um levantamento de iniciativas dos projetos do programa Meninas Digitais da Sociedade Brasileira de Computação.


%====LU NAO GOSTOU===
%Conforme exposto no Capítulo \ref{cap:Introducao}, nota-se que a evasão é diretamente ligada com a permanência nas universidades e quando analisa-se algum desses fatores é imprescindível observar o outro.  %confuso

%A realidade da evasão escolar no Brasil é abordada por diversos autores, bem como a participação das mulheres nos cursos das STEM, como descrito no Capítulo \ref{cap:FundamentacaoTeorica}. Também sabe-se que os incentivos a sua permanência ou ingresso nos cursos desta área não partem da legislação do país e geralmente contam com iniciativas provindas das próprias universidades. 

%Com isso, busca-se aqui compreender e analisar algumas destas iniciativas, trazendo assim sua devida importância, para que sirva de incentivo ou aprendizado para outros projetos e programas como o Meninas Digitais.
%====LU NAO GOSTOU FIM===


%Por isso esse trabalho 

%\section{Hipóteses}
%apenas para não esquecer...
%Será que pode ter um reflexo nas taxas de evasão pelo fato de ter um projeto?

\section{Objetivos}\label{sec:Objetivos}

Para organização do trabalho foram levantados objetivos gerais e específicos e estão apresentados nesta seção.

\subsection{Objetivo Geral}\label{sec:ObjGeral}
Analisar a presença das mulheres nos cursos de Computação e Tecnologias da Informação e Comunicação em universidades brasileiras e sua possível relação com as ações dos projetos parceiros do programa Meninas Digitais.

\subsection{Objetivos Específicos}\label{sec:ObjEspecifico}
Para alcançar o objetivo geral, tem-se como objetivos específicos:
\begin{itemize}

\item Investigar a questão da evasão e sua relação com gênero;

%\item Investigar a participação de mulheres na área das STEM para obter um panorama geral do tema;

\item Explorar as ações realizadas para diminuir a evasão das mulheres na área das STEM;

%\item Investigar os projetos brasileiros de incentivo a participação e presença de mulheres na área das STEM a fim de encontrar um programa estável;

\item Realizar um Mapeamento Sistemático da Literatura sobre evasão escolar;

\item Coletar e processar bases de dados relacionadas a evasão e gênero;

\item Entrevistar pessoas dos projetos parceiros do programa Meninas Digitais para coleta de dados.


%\item Coleta de dados por meio de entrevistas com pessoas dos projetos parceiros do programa Meninas Digitais;

%\item Analisar por frentes distintas a presença e evasão das mulheres nos cursos de TIC que possam ser afetadas por algum dos projetos de incentivo elencados;

%\item Investigar dentro dos projetos de incentivo elencados quais são suas principais iniciativas e frentes de trabalho;

%\item Aplicar e avaliar questionário com os projetos de incentivo elencados;

%\item Levantar todos os cursos de Computação e Tecnologias da Informação e Comunicação nas universidades públicas brasileiras;

%\item Analisar quais destes cursos brasileiros estão cadastrados no programa Meninas Digitais da Sociedade Brasileira de Computação;

%Identificar dez dos projetos cadastrados no programa Meninas Digitais que possuem as menores taxas de evasão escolar e maior presença feminina de acordo com os dados do INEP.

%\item Identificar dez cursos cadastrados no programa Meninas Digitais que possuem as menores taxas de evasão escolar e maior presença feminina de acordo com os dados do INEP.

%\item Análises gráficas sobre o corpo discente e docente, utilizando os dados do INEP dos dez cursos escolhidos;

%Dos dez cursos escolhidos, identificar a representatividade feminina do corpo docente;
%Analisar a partir de gráficos comparativos os  projetos selecionados de acordo com o primeiro objetivo;

%\item Aplicar questionário com mulheres participantes dos dez cursos escolhidos;

%\item Identificar ações de melhoria e padrões de atividades abordadas em cada projeto;
\end{itemize}

\section{Metodologia}\label{sec:Metodologia}
A presente pesquisa é classificada como quali-quantitativa, método misto que segundo \citeonline{creswell:2016} é definido como um
procedimento de coleta, análise e combinação de técnicas quantitativas e qualitativas em um mesmo desenho de pesquisa. A escolha deste método se dá pois a interação do qualitativo e do quantitativo traz mais possibilidades analíticas. O presente trabalho adota a definição de métodos mistos proposta por \citeonline{johnson:2004} que diz que tais métodos são “a classe de pesquisa em que o pesquisador mistura ou combina técnicas de pesquisa quantitativa e qualitativa, métodos, abordagens, conceitos ou linguagem em um único estudo” (JOHNSON; ONWUEGBUZIE, 2004: 17, tradução nossa).

Para a etapa de identificação e exploração do problema foi utilizado o procedimento metodológico de pesquisa bibliográfica para a estruturação dos conceitos e para obter uma visão geral da área.

Em seguida, para entendimento do cenário e estado da arte da predição da evasão escolar foi realizado um Mapeamento Sistemático da Literatura (MSL) que utilizou as diretrizes de \citeonline{petersen:2015}. 

Com o mapeamento ainda em andamento observou-se que havia a necessidade de um passo atrás. Observar não mais a predição da evasão escolar, mas sim as relações da evasão escolar com o engajamento e participação dos estudantes em projetos extracurriculares ou programas de incentivo a permanência. Momento em que a relação de evasão, gênero e programas de incentivo se relacionam e trazem a pesquisa este novo olhar, buscando por meio de análises quantitativas entendimento dos dados do INEP sobre essas perspectivas. 

Após as análises, por meio de entrevistas com os projetos selecionados busca-se encontrar as iniciativas realizadas, casos de sucesso e processos de abordagem para incentivo a permanência de mulheres na área da Computação e Tecnologias da Informação e Comunicação.

E por fim, analisar as possíveis relações entre a evasão ou permanência de mulheres nos cursos de TIC com o Programa Meninas Digitais. 

%XX, pois sua principal característica XXX.
%A primeira etapa consistiu em....
%Em seguida realizou-se um Mapeamento Sistemático da literatura, a fim de identificar melhor o campo de trabalho sobre a evasão, observando como é classificada e em quais contextos educacionais é mais observada.



\section{Organização do Texto}\label{sec:Estrutura}
A organização da pesquisa se dá em forma de capítulos e inicia com a fundamentação teórica, que visa descrever os principais tópicos do Capítulo \ref{cap:FundamentacaoTeorica} que são evasão escolar, dados do INEP e suas classificações, a participação de mulheres nas STEM e o cálculo da evasão escolar.

No Capítulo \ref{cap:trabRelacionados} apresentam-se os trabalhos relacionados a esta dissertação. Os trabalhos apresentados relacionaram intervenções ou projetos, como o Meninas Digitais a dados educacionais, como a nota dos estudantes, perfil e participação e apresentam propostas para amenizar a evasão ou aumentar a permanência de estudantes da área das STEM, em específico mulheres.

No Capítulo \ref{cap:Desenvolvimento} é apresentado o processo de pesquisa, onde inicialmente foi desenvolvido um Mapeamento Sistemático da Literatura para compreensão da evasão e dos algoritmos para sua predição e em seguida foi dado um passo atrás e se inicia o entendimento dos dados escolares do INEP relacionando-os com o Programa Meninas Digitais. Nesta etapa os dados são explorados e analisados a partir de gráficos e tabelas. Propõe-se também realização de entrevistas com as pessoas de projetos parceiros do Programa Meninas Digitais escolhidos para melhor entendimento das propostas, colaborações e incentivos a permanência de mulheres nos cursos de Computação e Tecnologias da Informação e Comunicação.

O Capítulo \ref{cap:Consideracoes} apresenta um cronograma de atividades e os resultados parciais deste trabalho.


