\chapter{Considerações}\label{cap:Consideracoes}
O principal objetivo deste trabalho está na análise da presença das mulheres nos cursos de TIC das universidades brasileiras e a busca por uma possível relação com as ações dos projetos parceiros do programa Meninas Digitais da SBC. 

Para encontrar esta possível relação inicialmente fez-se uma investigação e pesquisa exploratória sobre a evasão escolar e sua relação com gênero e uma exploração de ações realizadas para diminuir a evasão das mulheres na área das STEM.

Com a realização do Mapeamento Sistemática da Literatura, foi possível identificar e reformular o objetivo deste trabalho. Foi possível também trazer algumas informações importantes, como as definições de evasão e as principais variáveis consideradas para sua previsão.

Após todas as definições e investigações, fez-se a coleta de dados relacionados a evasão e gênero, tornando possível as análises. As análises envolvem características do estudante e para a próxima etapa de trabalho, pretende-se explorar melhor todas estas informações, trazendo também outras perspectivas ainda não abordadas.

Diante disso, o trabalho apresenta-se em construção. O cenário da presença feminina nos cursos de TIC foi explorado parcialmente e as entrevistas, assim que elaboradas, são promissoras para a efetiva busca do objetivo desta dissertação.


\section{Cronograma}\label{sec:Cronograma}
A fase atual da pesquisa é a elaboração e validação das perguntas da entrevista.

\begin{table}[h]
\centering
\caption{Cronograma de Atividades}
\label{tab:ProjetosMeninasDigitais}
\resizebox{\textwidth}{!}{%
\begin{tabular}{|l|l|l|l|l|l|l|}
\hline
Atividade                                          & 07/2022                  & 08/2022                  & 09/2022                                         & 10/2022                  & 11/2022                  & 12/2022                  \\ \hline
Validação das perguntas        & \cellcolor[HTML]{C0C0C0} &                          &                                                 &                          &                          &                          \\ \hline
Contato com selecionados                &                          & \cellcolor[HTML]{C0C0C0} &                                                 &                          &                          &                          \\ \hline
Aplicação da entrevista         &                          & \cellcolor[HTML]{C0C0C0} &                                                 &                          &                          &                          \\ \hline
Análises finais dos dados do INEP                 &                          &                          & \cellcolor[HTML]{C0C0C0}{\color[HTML]{000000} } &                          &                          &                          \\ \hline
Análise dos dados das entrevistas &                          &                          &                                                 & \cellcolor[HTML]{C0C0C0} &                          &                          \\
\hline
Triangulação dos dados                            &                          &                          &                                                 &                        \cellcolor[HTML]{C0C0C0}  & \cellcolor[HTML]{C0C0C0} &                          \\
\hline
Escrita da dissertação                            &                         \cellcolor[HTML]{C0C0C0} &                         \cellcolor[HTML]{C0C0C0} &                                                \cellcolor[HTML]{C0C0C0} &                        \cellcolor[HTML]{C0C0C0}  & \cellcolor[HTML]{C0C0C0} &                          \\ \hline
Defesa da dissertação                             &                          &                          &                                                 &                          &                          & \cellcolor[HTML]{C0C0C0} \\ \hline
\end{tabular}%
}
\end{table}

\section{Resultados Parciais}\label{sec:ResultadosParciais}
 Este projeto de pesquisa em desenvolvimento apresenta alguns resultados parciais como:
 As análises gráficas a partir dos dados do INEP foram organizadas em um site público e como fruto desta produção tem-se a publicação do artigo intitulado 'Ferramenta de Visualização de Dados do Censo da Educação Superior do INEP' que foi aceito para ser apresentado no Congresso da SBC de 2022 no 10º Workshop de Computação Aplicada em Governo Eletrônico. O MSL está em processo final de produção para submissão.

Além disso também foi produzido e submetido na Revista Novas Tecnologias na Educação o artigo intitulado 'Análise da evasão Feminina nos Cursos de Ciência da Computação das Universidades Públicas e Presenciais de Santa Catarina' que apresenta uma análise dos dados do INEP do ensino superior, com foco nos cursos de bacharelado em Ciência
da Computação nas universidades de ensino presencial e gratuito do estado de Santa Catarina, abrangendo os anos de 2015 a 2019. O trabalho tem como
objetivo apresentar as relações comparativas entre gênero, evasão e demais fatores impactantes como raça, forma de ingresso e idade. Como resultado, o artigo apresenta uma grande disparidade nas taxas, observando a maior evasão em ambos os gêneros acima de 35 anos.
%falar das producoes e artigos submetidos..
