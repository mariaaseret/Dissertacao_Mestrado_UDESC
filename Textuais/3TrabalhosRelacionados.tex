\chapter{Trabalhos Relacionados}
%%amadurecer e ver o que esses trabalhos se relacionam com o meu trabalho.
\label{cap:trabRelacionados}
Por meio de um levantamento bibliográfico da literatura buscou-se identificar trabalhos relacionados que abordem as áreas  deste trabalho: evasão, programas de incentivo a permanência e aumento da presença feminina nos cursos de Computação e Tecnologias da Informação e Comunicação. 

São descritas ações diversas, a Seção \ref{sec:impulsoparamulheres} apresenta o programa de impulso para mulheres na área da Ciência da Computação, mostrando iniciativas como cursos, grupos de suporte e entre outros que por fim traz números positivos na permanência de mulheres na área. 

A Seção \ref{sec:intervencaoinclusaoretencao} apresenta uma intervenção a fim de amenizar a diferença de gênero no curso de Ciência da Computação. Propõe modificar a forma com que os estudantes recebem suas notas e após essa intervenção apresenta suas descobertas positivas. 

A Seção \ref{sec:meninasuff} apresenta uma iniciativa brasileira de um projeto parceiro do programa Meninas Digitais que propõe uma dinâmica com 37 estudantes e por fim propõe um debate sobre o motivo do baixo número de mulheres na área das TICs.

Na Seção \ref{sec:subareas}, as autoras mostram a relação de gênero e a presença de mulheres dentro das sub-áreas da computação, trazendo assim outra perspectiva da relação de gênero, mesmo assim contribuem para a construção deste trabalho.

O trabalho apresentado na Seção \ref{sec:IniciativasPermanencia} mostra uma revisão sistemática da literatura, que seleciona 73 artigos para encontrar o que a literatura apresenta a respeito de iniciativas educacionais para motivar a permanência das mulheres nas universidades brasileiras nos cursos relacionados à Computação na última década (2010-2019) e como conclusão elenca uma série de iniciativas encontradas.%TERMINAR COM OS OUTROS **

%De acordo com o Censo da Educação Superior de 2015, as mulheres foram a maioria entre as matriculas ativas em cursos da educação superior no Brasil, representando cerca de 55,6\% de todos os estudantes naquele ano. Além disso, também foram a maioria entre os ingressantes (53,9\%) e os concluintes (59,9\%). Porém, ao analisar os cursos escolhidos por homens e mulheres, é notável a diferença entre as áreas escolhidas, pois em sua grande maioria, as mulheres acabam optando por cursos relacionadas ao cuidado, como Pedagogia, Enfermagem e Psicologia, já os homens acabam escolhendo cursos mais relacionados com as áreas tecnológicas, como as Engenharias e os cursos de Computação \cite{moreira:2018}.

\section{Um Impulso para Mulheres na Ciência da Computação}\label{sec:impulsoparamulheres}
Nesta Seção o autor \citeonline{beck:2007} apresenta uma iniciativa que aconteceu em sua Universidade \textit{Truman State}. Tal projeto teve como inspiração outra iniciativa que obteve sucesso na retenção de mulheres na computação: um grupo de suporte para mulheres da Universidade de \textit{Carnegie Mellon}, \textit{o Carnegie Mellon Women’s Association}.

O autor \citeonline{beck:2007} apresenta um projeto iniciado no ano de 2002 de sua universidade que contou com um incentivo financeiro inicial de 40.000 dólares e foi intitulado “\textit{A Jumpstart for Women in CS}” (Um Impulso para Mulheres na Ciência da Computação). 

Este projeto buscou diversas iniciativas para a inclusão das mulheres e maior familiaridade com o curso e ter contato e inspirações de outras mulheres. Suas principais iniciativas propostas foram um acampamento de duração de duas semanas, antes do início do curso, para impulsionar as calouras; um programa de mentoria realizada por mulheres veteranas no curso para auxiliarem as calouras durante o ano inicial, o qual é o mais complicado em função da adaptação necessária; um projeto para convidar mulheres bem sucedidas na área para palestrarem sobre suas experiências e apresentarem as oportunidades disponíveis para as mulheres na área; além de um grupo de suportes fornecendo um local para a realização de encontros para todas as estudantes de ciência da computação, um ambiente para mulheres debaterem e realizarem perguntas sobre assuntos, as quais normalmente gerariam desconforto se feitas em público e além de um ambiente para encorajar mulheres a realizarem pesquisas sobre tópicos relevantes para elas.

Todas estas iniciativas surgiram a partir das problemáticas de gênero nos cursos de tecnologia, \citeonline{beck:2007} em seu estudo, comenta sobre os paradigmas sociais presentes nas salas de aula dos cursos de Ciência da Computação, onde é possível observar que em sua grande maioria, que os estudantes homens costumam desde o início, mexer, montar e desmontar computadores com mais frequência, além de muito se auto titularem como “\textit{hackers}”, “\textit{computer nerds}” ou “\textit{geeks}”. Já as mulheres, por sua vez, não costumam se titular de tal forma, pois geralmente se preocupam mais com as aplicações sociais e de resolução de problemas da computação. Segundo o autor, essa situação tende a gerar consequências, fazendo com que as estudantes se sintam deslocadas desde o início do curso, gerando incertezas e sentimentos como não saberem tanto quanto deveriam. Com isso, muitas mulheres acabam se desencorajando ao receberem as primeiras notas baixas, o que é normal entre quaisquer estudantes no início da computação, assim tomando isso como prova de seus medos.

Como resultados \citeonline{beck:2007} mostra que o projeto foi um sucesso para as mulheres envolvidas, 3 das 5 calouras que compuseram o grupo que participou do primeiro acampamento completaram o curso de bacharelado em Ciência da Computação. Considerando as mentoras envolvidas no projeto, chega-se a um total de 8 mulheres de 7 que acabaram conseguindo se formar no curso. 

Além das primeiras 8 participantes mencionadas, nos 5 anos desde o início do projeto, 91\% das mulheres graduandas de Ciência da Computação concluíram seus estudos na área o que em comparação com as mulheres que não participaram do projeto, aproximadamente 75\% finalizaram o curso de Ciência da Computação. \citeonline{beck:2007} alega nesta forte conclusão que nenhuma relação de causa e efeito está sendo feita, mas diz que a correlação é forte e certamente a participação no projeto tem um impacto significativo para as mulheres que obtém seus diplomas em Ciência da Computação.



\section{Intervenção para maior Inclusão e Retenção de Mulheres na Ciência da Computação}\label{sec:intervencaoinclusaoretencao}
\citeonline{fisk:2021} inicialmente apresentam um panorama das mulheres nos cursos de Ciência da Computação das universidades de seu país, Estados Unidos. Em seu trabalho considera 
os esforços para a inclusão e retenção e apresenta números do ano de 2016 dos dados fornecidos pela \textit{National Science Foundation} onde apenas 19\% dos diplomas são de mulheres. Ainda defendem que tal situação possivelmente acontece em função dos esteriótipos sociais presentes na área, que fazem as mulheres não se sentirem capazes o suficiente para desempenhar suas habilidades em um contexto predominantemente dominado pelos homens, o que faz desencorajá-las em relação a continuidade de carreira na área.

\citeonline{fisk:2021} relatam que nas fases iniciais dos cursos de Ciência da Computação, normalmente os estudantes apresentam baixos desempenhos nas disciplinas, assim não alcançando as notas desejadas pelos mesmos, o que acaba afetando a persistência dos estudantes no curso, ainda mais em mulheres que já duvidam de suas capacidades em função dos esteriótipos presentes nas salas de aula. Ainda segundo os autores, um \textit{feedback} mais detalhado sobre as avaliações realizadas pelos estudantes pode auxiliar a diminuir a diferença entre homens e mulheres no curso.

Após o relato feito pelos autores como uma tentativa de amenizar essas diferenças, realizou-se uma leve intervenção que aconteceu durante dois semestres em uma disciplina do curso de Ciência da Computação da Universidade de Carolina do Norte nos EUA. Tal intervenção contou com a participação de 193 estudantes, os quais foram divididos em dois grupos, sendo que os estudantes que compunham o grupo de controle, apenas receberam suas notas por e-mail, como já acontecia normalmente, enquanto os estudantes que estavam no grupo de intervenção, além de receberem suas respectivas notas por e-mail, também receberam informações contextuais extras, como por exemplo, falas dos professores sobre seus desempenhos e que os estudantes eram capazes de persistirem no curso.

Para a avaliação dos resultados sobre a intervenção realizada, \citeonline{fisk:2021} entrevistaram os estudantes participantes antes e depois de todo o processo, principalmente em relação ao sentimento de persistirem no curso, fazendo com que a pesquisa traga e aplique a teoria sociológica (sobre processos sociopsicológicos de gênero em torno de autoavaliações de habilidade e escolha de carreira) à pesquisa em educação em computação, além disso o projeto examinou como as autoavaliações da habilidade de estudantes de Ciência da Computação contribuem para intenções de persistência no curso e também avaliou a intervenção leve projetada para aumentar as intenções de persistência das mulheres na área.


Como descobertas, \citeonline{fisk:2021} apresenta que a intervenção leve proposta pelo projeto aumentou a capacidade de autoavaliação dos estudantes de Ciência da Computação para os homens e as mulheres e aumentou as intenções de persistência no curso das estudantes mulheres. Além disso encontrou que a melhora na capacidade autoavaliada das mulheres do curso de Ciência da Computação pode aumentar suas intenções de persistência no curso.

Para chegar nessas descobertas mencionadas realizou entrevistas com os estudantes antes e depois da intervenção, porém as perguntas desse processo não foram mencionadas ao decorrer do artigo. 


\section{\#Include <meninas.uff>}\label{sec:meninasuff}
Em um contexto brasileiro, o trabalho realizado por \citeonline{mochetti2016ciencia}, pesquisadoras da Universidade Federal Fluminense (UFF), ressaltam a importância de haver igualdade em todas as áreas do conhecimento, pois assim todos os indivíduos acabam sendo representados, possibilitando o desenvolvimento de soluções de forma adequada para os inúmeros problemas presentes em nossa sociedade. Porém, ao analisarem os dados fornecidos pela Pesquisa Nacional por Amostra de Domicilio (PNAD), no ano de 2009 as mulheres consistiam apenas 18,84\% entre todos os profissionais de TI no Brasil. 

As autoras abordam que as dificuldades encontradas na inclusão de mulheres em ambientes predominantemente dominados por homens é um problema antigo, demonstrando como a presença da mulher na sociedade é regida por regras masculinas. Como forma de estudar e tentar diminuir este problema, existem diversos congressos na área de TIC, como os citados pelas autoras: \textit{Grace Hopper Celebration of Women in Computing do Instituto Anita Borg} e a \textit{Association for Women in Mathematics}. Outra forma de incentivo de mulheres nas áreas de TIC que as autoras mencionam, é sobre a criação de diversos programas nacionais, e citam também o Programa Meninas Digitais da SBC.

Priorizando uma pesquisa e o acompanhamento dos resultados de um programa pertencente ao Instituto de Computação da UFF, chamado de \#include <meninas.uff>, as autoras relatam que entre os 70 inscritos em 2016 no curso de Ciência da Computação, somente 8 eram mulheres. Com isso, descrevem uma atividade com o principal objetivo de analisar o comportamento entre estudantes e homens e mulheres do curso, além da realização de debates relacionados a isso.

%talvez colocar sobre os dados históricos aqui

A atividade proposta foi dividida em três partes principais, e contou com a participação de duas professoras e três estudantes da graduação como observadoras. A primeira etapa foi para realizar o recrutamento dos participantes, sendo que não foi informado aos estudantes que a atividade seria sobre a falta das mulheres na área de TI. Com a primeira etapa concluída, os 37 estudantes recrutados, sendo apenas 5 mulheres, foram dispostos em um círculo para que todos se apresentassem para os demais. 

Já a segunda etapa consistiu em uma dinâmica em grupo, na qual os estudantes foram divididos em 5 grupos, sendo que cada um continha uma mulher. Após a divisão, foi distribuído um número para cada menina e em seguida um número para cada menino, e então os meninos deveriam encontrar as meninas que possuíssem o mesmo número, tal ação foi feita pelas pesquisadoras com a intenção de indiretamente dar as mulheres um papel de liderança. Após a formação dos grupos concluída, todos tiveram que desvendar dicas de uma caça ao tesouro feita com base na Cifra de Cesar (Um tipo de cifra de substituição na qual cada letra do texto é substituída por outra), a intenção dessa atividade era analisar o comportamento entre os integrantes de cada grupo, além de observar se as meninas continuariam exercendo seus papéis de liderança.

Por fim a terceira e última etapa da atividade, serviu para todos os estudantes realizassem um debate aberto sobre os motivos do número de mulheres na área de TI estar diminuindo cada vez mais, e o porque desse número ser tão baixo, além de fornecer as meninas participantes a oportunidade para que todas compartilhassem suas próprias histórias pessoais e seus motivos de ingressarem em um curso dominado por homens. 

Como resultado da atividade, mostrou como é importante o trabalho não apenas com mulheres, mas também com os homens. Mostram também que criar um debate com a participação masculina pode ajudar na conscientização do papel pressor que exercem como maioria e isso por si só pode diminuir a hostilidade do meio e atrair mais meninas e reduzir a desistência por parte das estudantes. 


\section{Mulheres na Computação: Análises por Sub-Áreas }\label{sec:subareas}

As análises da presença de mulheres nas áreas de STEM podem tomar outros rumos, esta questão também é observada de outras perspectivas que não sejam durante a graduação. No trabalho de \citeonline{duarte:2019} as autoras observam a presença feminina em Comitês de Programa (CP) de simpósios da Sociedade Brasileira de Computação (SBC). \citeonline{duarte:2019} da Universidade Federal de Minas Gerais (UFMG), propõe uma metodologia diferenciada para caracterizar a diversidade de gêneros em sub-áreas da Computação neste contexto. Segundo as autoras, a principal motivação para proporem tal metodologia, é pelo fato de que os CPs são a melhor amostra possível da parte mais ativa da comunidade de uma determinada sub-área, além de que os membros presentes têm o poder de decisão sobre quais artigos serão aceitos e publicados, assim exercendo uma função mais decisória na comunidade.

O estudo relatado no trabalho, baseia-se em dados providos de 17 eventos nacionais, e conduz as seguintes questões: como é a presença feminina por evento e como a mesma evolui no tempo; e como as mulheres participam de comitês de eventos diferentes. Ainda é mencionado que as iniciativas brasileiras voltadas para o incentivo das mulheres nas áreas de computação, podem usufruir dos resultados obtidos no presente estudo, possibilitando a análise de quais são as áreas profissionais em que as mulheres estão mais presentes.

Nos trabalhos relacionados, as autoras comentam que há muitas iniciativas que visam atrair as mulheres para a Computação, sendo que muitas dessas iniciativas são apresentadas durante o evento do WIT que ocorre anualmente, além de eventos internacionais, como por exemplo, o \textit{Stanford Artificial Intelligence Laboratory’s Outreach Summer} (SAILORS), que tem como foco uma sub-área específica da Computação, a Inteligência Artificial. Com isso, levantam a seguinte questão: quais seriam as outras sub-áreas da Computação mais convidativas para atrair meninas?

A resposta para a questão levantada, é encontrada a partir da caracterização das diversas áreas presentes no contexto nacional, assim sendo possível a identificação de aspectos que podem contribuir para atratividade que as áreas proporcionam às mulheres. 

Os resultados obtidos a partir das análises realizadas sobre a presença das mulheres em comitês dos simpósios providos pela SBC, indicam que o público feminino está mais presente em eventos de áreas multi-disciplinares, os quais são relacionados sobre assuntos como a aplicação da computação ou com aspectos sociais. Por fim, as autoras ainda comentam sobre as futuras atividades a serem realizadas, voltadas para um melhor entendimento e exploração sobre as características das áreas femininas. 


\section{Iniciativas Educacionais para Permanência das Mulheres em Cursos de Graduação em Computação no Brasil}\label{sec:IniciativasPermanencia}

O artigo escrito pelas pesquisadoras \citeonline{holanda:2020} da Universidade de Brasília (UnB) é introduzido com a constatação da baixa representatividade das mulheres em cursos de Computação.

%Ainda mencionam que entre os anos de 2000 e 2013, o número de homens que concluíram o curso de Ciência da Computação aumentou 98\%, enquanto que o de mulheres decresceu em 8\%. Já na UnB durante o ano de 2019, entre todos os ingressantes em cursos do Departamento de Computação, a presença feminina foi inferior a 20\%.

Além disso, o artigo apresenta diversas iniciativas edicionais presentes no Brasil, as quais que têm como propósito de lidar com a relação entre as mulheres e os cursos de computação, baseando se em um protocolo de revisão sistemática, e com a seguinte questão de pesquisa: “O que a literatura apresenta a respeito de iniciativas educacionais para motivar a permanência das mulheres nas universidades brasileiras nos cursos relacionados à Computação na última década (2010-2019)?”. Enfatizando assim as análises dos programas que visam incentivar a permanência das mulheres nos cursos de Computação.

Considerando todas as baixas taxas da presença feminina entre os cursos relacionados as áreas de TI, as pesquisadoras mencionam inicialmente o programa Meninas Digitais da SBC, o qual foi criado com principal intuito de incentivar e encorajar as mulheres a seguirem as carreiras em computação. Afirmam que o programa já contava em 2020 com 110 projetos espalhados por todo o país.

Para responder esta pergunta foi utilizada a metodologia baseada no processo de revisão
sistemático da literatura baseado no protocolo definido por \citeonline{kitchenham:2009}. Após aplicarem todo o processo de inclusão e exclusão, foram utilizados 73 artigos para responder as questões de pesquisa (QP) 1,2,3 e 4 e 28 artigos para responder a QP 5, com ações educacionais em universidades brasileiras. Desta forma, foram derivadas cinco QP:

%A metodologia constitui-se com o mapeamento dos documentos acadêmicos existentes para obter a resposta da questão principal de pesquisa. Para facilitar a resposta da questão principal, outras cinco perguntas apoiaram a pesquisa e são descritas a baixo:

\begin{itemize}
    \item QP1: Como encontram-se distribuídos por ano os documentos acadêmicos?
    \item QP2: Em quais outras conferências, além do WIT-SBC e LAWCC-CLEI, os pesquisadores brasileiros publicam seus artigos neste tema?
    \item QP3: Quais revistas os pesquisadores brasileiros escolhem para publicar sobre esse tema?
    \item QP4: Em quais estados brasileiros residem os pesquisadores que mais publicam?
    \item QP5: Quais são as atividades educacionais para as mulheres em cursos de graduação em Computação?
\end{itemize}

Como resposta da QP1 observaram que o número de artigos relacionados as alunas da graduação de Ciência da Computação do Brasil cresceu de forma significativa a partir do ano de 2016, sendo que foi no ano de 2017 que mais houveram publicações relacionadas ao tema, com 18 documentos, isto deve se muito pelo fato que este foi o primeiro ano que o Womens in Technology (WIT) realizou chamadas e publicações de artigos. Porém muitos desses artigos apenas abordam os problemas envolvendo a falta de mulheres nos cursos de computação, não apresentando possíveis ações ou intervenções para melhorar tal cenário de desigualdade.

Para responder a QP2, observaram que os eventos com mais artigos publicados sobre o tema são o WIT-SBC e o LAWCC-CLEI, com 39 e 15 artigos, respectivamente. Porém também foram encontradas publicações em Rede Feminina Norte e Nordeste de Estudos e Pesquisas sobre Mulheres e relações de Gênero (REDOR), Congresso Brasileiro de Informática na Educação (CBIE), Culturas, Alteridades e Participações em IHC (CAPA IHC) e Workshop sobre Educação em Computação (WEI-SBC) com respectivamente 2, 3, 1 e 2 artigos. Já em relação a Ações Educacionais (AE) na graduação para mulheres, foram encontrados somente artigos no WIT e no Congresso da Mulher Latino-americana em Computação (LAWCC), como 21 e 5 artigos respectivamente. 

Já a QP3 apontam que entre todos os 73 artigos com o tema de mulheres em cursos de graduação em Ciência da Computação encontrados, 11 artigos foram publicados em 9 revistas acadêmicas, sendo estas: DADOS-Revista de Ciências Naturais, Revista de Estudos Feministas, Cadernos Pagu, Vivências: Revista Eletrônica de Extensão da URI e Revista Diversidade e Educação. Porém, com atividades educacionais, apenas dois artigos foram encontrados, nas revistas Revista da Escola Regional de Informática e Vivências: Revista Eletrônica de Extensão da URI. 

Como resultado da QP4, mostram que os quatro estados que mais publicaram artigos sobre o tema de mulheres em cursos de graduação em Ciência da Computação, foram respectivamente: Amazonas com 10 artigos, os quais foram produzidos por autoras da Universidade Federal do Amazonas e da Universidade do Estado do Amazonas. Em seguida ficou São Paulo com 7 artigos, produzidos por autores das seguintes instituições: Universidade de São Paulo, Universidade Estadual de Campinas, Instituto Federal de Educação, Ciência e Tecnologia São Paulo e Universidade Federal de São Paulo. Por fim, Paraíba com 7 artigos, os quais foram produzidos por autores da a Universidade Federal da Paraíba e o estado do Rio Grande do Sul com também 7 artigos, que foram produzidos por autores das seguintes instituições: s Universidade Federal do Rio Grande do Sul, Universidade de Passo Fundo, Universidade Federal do Pampa, Universidade Regional Integrada do Alto Uruguai e das Missões e Universidade Federal do Rio Grande. Contudo, sobre os artigos com atividades educacionais, os estados que mais publicaram foram Amazonas e Paraíba com 9 e 3 documentos respectivamente.

Para responder a QP5, \citeonline{holanda:2020} utiliza-se dos 28 artigos que de fato focam em ações realizadas para motivar alunas a persistirem nos cursos. Os artigos levantados citam atividades como: palestras com profissionais da área de computação, minicurso de programação web e oficina de programação em Python — IFSP, projeto IF(meninas){nas exatas}; oficinas que trabalham conceitos de robótica e Arduíno, palestras com foco no aumento feminino na ciência e também tratando-se da presença histórica de mulheres na tecnologia — Instituto Federal Goiano; atividades de competição como programação competitiva; hackathons. Algumas das atividades citadas, como hackathons, foram realizadas pelo projeto SciTechGirls, e outras, como programação competitiva, pelo projeto Cunhatã Digital. Entre todas as atividades educacionais, a maioria provém da Universidade Federal do Amazonas e a Universidade do Estado de Amazonas, além também da Universidade Federal da Paraíba.

A partir do que foi abordado ao longo do artigo, é possível observar que o Brasil tem trabalhado bastante em prol da inclusão das mulheres na Computação em quaisquer dos níveis educacionais. Considerando somente o contexto da educação superior, maior parte dos artigos demonstra a baixa quantidade de mulheres presentes em cursos relacionadas a Computação. Ainda é abordado que o caminho é longo para mitigar este problema, mas ressaltam que o Brasil tem trabalhado para mudar esse cenário de desigualdade, agindo desde o ensino fundamental até o nível superior.



\section{Discussão do Capítulo}\label{sec:DiscussaoTrabRel}

%falar ponto principal e ligar com o próximo capitulo **
O capítulo \ref{cap:trabRelacionados} apresenta 5 trabalhos que relacionam-se com a presente dissertação. Todos os artigos abordados nas seções, com exceção da Seção \ref{sec:subareas} e \ref{sec:IniciativasPermanencia} apresentam análises sobre dados educacionais como evasão, presença de mulheres, notas e participação e relacionam com iniciativas para a permanência de mulheres em cursos de TIC. 

O trabalho presente na Seção \ref{sec:subareas}, diferente dos outros apresentados, também traz um estudo sobre a presença das mulheres na área das STEM e iniciativas na participação, porém ao invés de um contexto de estudantes da graduação, apresentam a relação de gênero nas sub-áreas da computação.

Já o trabalho apresentado Seção \ref{sec:IniciativasPermanencia}, faz uma revisão sistemática da literatura com o intuito de mapear quais são esses programas de incentivo ou permanência de mulheres na área das TICs, mostrando que o Brasil mostra-se ativo comparado com outros países.

Desta forma a proposta de pesquisa apresentada no próximo capítulo envolve análises de dados da educação superior brasileira e o cruzamento de informações dos projetos do Programa Meninas Digitais, em busca de possíveis relações com as ações dos projetos
parceiros do programa Meninas Digitais e a evasão.